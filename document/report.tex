\documentclass{article}
\usepackage{amsmath}
\usepackage{bigints}
\begin{document}
1. Conservation of the total relative concentration $\Phi$ (Our order parameter)

Just like the total number of particles, $\Phi$ is conserved.
To demonstrate so, we start by reminding ourselves that $\phi(\mathbf{x},t) = c_A(\mathbf{x},t)-c_B(\mathbf{x},t)$.
Whence \[ \Phi(t) = \int_{\Omega} \phi(\mathbf{x},t)d\mathbf{x} = N_A - N_B, \]
where $\Omega$ is the available space.

$\Phi$ is thus conserved, since $N_A$ and $N_B$ also are over time.

Finally, we define as an error metric the proportional change in this quantity.
Numerically, we just calculate the sum of the relative concentration over the whole spatial grid.
We define
\[ \epsilon(t) = log_{10}(\frac{\Phi(t)}{\Phi_0}-1) \]
We take the logarithm since the scales are very small $(\epsilon(t)<-14)$.
%The second-order term is dependent on the squared 2-norm of the gradient of the concentration.
%That is, it accounts for the increment in the free energy due to the heterogeneity in the system (Higher variation in composition, higher concentration gradient).
%For a high epsilon value, to minimize the Helmholtz free energy, we must make the second order term smaller overall (tendence towards zero).
%This gives rise to a weaker phase separation and thus results in a more homogeneous, miscible, monophasic fluid
%In our simple model, this term accounts for the spinodal decomposition we observe in fluids.
\end{document}
