\documentclass{article}
\usepackage{amsmath}
\usepackage{bigints}
\begin{document}
1. Conservation of the total relative concentration $\Phi$ (Our order parameter)

Just like the total number of particles, $\Phi$ is conserved.
To demonstrate so, we start by reminding ourselves that $\phi(\mathbf{x},t) = c_A(\mathbf{x},t)-c_B(\mathbf{x},t)$.
Whence \[ \Phi(t) = \int_{\Omega} \phi(\mathbf{x},t)d\mathbf{x} = N_A - N_B, \]
where $\Omega$ is the available space.

$\Phi$ is thus conserved, since $N_A$ and $N_B$ also are over time.

Finally, we define as an error metric the proportional change in this quantity.
Numerically, we just calculate the sum of the relative concentration over the whole spatial grid.
We define
\[ \epsilon(t) = log_{10}(\frac{\Phi(t)}{\Phi_0}-1) \]
We take the logarithm since the scales are very small $(\epsilon(t)<-14)$.
\end{document}
